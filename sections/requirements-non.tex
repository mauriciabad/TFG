\clearpage\newpage
\section{Non-Functional requirements}
\label{sec:non-functional}

% \subsection{For the students}
% \begin{multicols}{3}
% \noindent
% Be \textbf{reliable}:
% \vspace{-5mm}
% \begin{itemize}[leftmargin=*]
%     \setlength\itemsep{-1em}
%     \item Don't lose data.
%     \item Don't miscalculate.
%     \item Keep grades private.
%     \item Work as expected.
%     \item Work on all devices.
%     \item Work offline.
% \end{itemize}
% \columnbreak
% \noindent
% Be \textbf{easy to learn} and use:
% \vspace{-5mm}
% \begin{itemize}[leftmargin=*]
%     \setlength\itemsep{-1em}
%     \item Simple and attractive UI.
%     \item Feel fast.
%     \item Place elements where expected.
%     \item Shows clearly where results come from.
% \end{itemize}
% \columnbreak
% \noindent
% Be \textbf{engaging}:
% \vspace{-5mm}
% \begin{itemize}[leftmargin=*]
%     \setlength\itemsep{-1em}
%     \item Get updates.
%     \item Have "cool" details.
%     \item Encourage to share the app.
%     \item Have an easy to remember name, logo, and domain.
% \end{itemize}

% \end{multicols}

% \subsection{For the developers}

% \begin{itemize}
%     \setlength\itemsep{-1em}
%     \item Have the minimum economic costs, zero if possible.
%     \item Use modern technologies.
%     \item Generate some economic income to pay for expenses.
%     \item Use open source technologies.
% \end{itemize}

This section highlights the 4 non-functional requirements (NFR) that the application must meet. These requirements shape the app so much that all features aim to satisfy them.

\subsection*{NFR 1 - Reliable}
\begin{itemize}
    \item \textbf{Description}: The functionalities have to work consistently and predictably, in other words, the app has to work as the user expects. For example, it doesn't lose data, it doesn't miscalculate, it works on all the user's devices, it works offline...
    \item \textbf{Justification}: The app has to work properly to be useful, otherwise there's no point in using it.
    \item \textbf{Acceptance criteria}: Ask a sample of users to evaluate, from 1 to 5, how much they agree to the following statements. A mean greater than 2.5 needs to be achieved.
    \begin{itemize}[noitemsep]
        \item The app works as I expect.
        \item The app does the calculations without mistakes.
        \item I can use the app from any of my devices.
        \item The app has never lost any of my data.
        \item I can use the app with the subjects I course.
    \end{itemize}
\end{itemize}

\subsection*{NFR 2 - Easy to use}
\begin{itemize}
    \item \textbf{Description}: The app has to be intuitive and accessible, in other words, easy to learn and easy to use.
    \item \textbf{Justification}: If the app is easy to use the users will save time and overall be more satisfied.
    \item \textbf{Acceptance criteria}: Ask a sample of users to evaluate, from 1 to 5, how much they agree to the following statements. A mean greater than 2.5 needs to be achieved.
    \begin{itemize}[noitemsep]
        \item The app is easy to learn
        \item The app is easy to use
        \item The app responds quickly to my actions.
        \item I know how to use the app in general.
        \item I know how to create a subject
        \item I understand all the information in the dashboard.
    \end{itemize}
\end{itemize}

\subsection*{NFR 3 - Engaging}
\begin{itemize}
    \item \textbf{Description}: The app must engage users and ensure that they are enjoying it and will recommend it.
    \item \textbf{Justification}: Happy users will attract new users helping the app grow. If the users are engaged, that means that the app is useful to them.
    \item \textbf{Acceptance criteria}: Ask a sample of users to evaluate, from 1 to 5, how much they agree to the following statements. A mean greater than 2.5 needs to be achieved.
    \begin{itemize}[noitemsep]
        \item I enjoy using the app.
        \item I would recommend the app.
        \item There's people using the app thanks to me.
        \item The app helps me to achieve my goals.
        \item The app helps me to pass the subjects.
        \item I open the app frequently.
        \item I open the app when I receive a new grade.
        \item GradeCalc's logo, name, and style are easy to remember.
        \item I like GradeCalc's logo, name, and style.
    \end{itemize}    
\end{itemize}

\subsection*{NFR 4 - Cheap to produce}
\begin{itemize}
    \item \textbf{Description}: The app has to be as cheap as possible to develop and maintain.
    \item \textbf{Justification}: This is a small hobby project developed by a student that is not willing to spend money on it unless it's imperceptible or would make a big difference.
    \item \textbf{Acceptance criteria}: the app must have a profit greater than -50€/year and cost less than 100€ to develop.
\end{itemize}

