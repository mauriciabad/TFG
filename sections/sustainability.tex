\section{Sustainability}
\subsection{Self-assessment}
After answering the sustainability analysis survey, I've realized that in any project there is a sustainability component divided into three different dimensions: economic, environmental, and social. It has also made me think about the causes, consequences, and possible solutions about social, environmental, and economical problems that affect all kinds of IT projects.
I also contemplated the importance of suggesting ideas and applying solutions to this project to make it more sustainable.

Not only I chewed over the importance of sustainability in a project like this one, but also I realized the importance I was giving to this topic was not enough in my past projects, as result of a little knowledge about topics like the environmental costs that IT products have over their lifespan. I neither knew how to measure the environmental impact of Information and communications technology in the world. And I also didn't know the existence of sustainable technologies applicable to a software project.
I already knew the importance of introducing social justice, equity, diversity, and transparency in IT projects.

To conclude, this project has made me reflect on the
the importance of taking into account sustainability when developing a project from all perspectives; economically, environmentally, and socially.

\subsection{Economic Dimension}
\noindent \textbf{Regarding PPP: Reflection on the cost you have estimated for the completion of the project}

The estimated cost of the project only includes essential resources, in this case, 0. I'm going to reuse all the hardware I currently own and sporadically use my friends' devices to test the app.


\noindent \textbf{Regarding Useful Life: How are currently solved economic issues (costs...) related to the problem that you want to address (state of the art)?}

All current solutions have a similar architecture, all of them are websites or apps without a database or a really simple one.
So, their costs must be the domain registration and hosting.

\noindent \textbf{Regarding Useful Life: How will your solution improve economic issues (costs ...) with respect to other existing solutions?}

I'm going to use many free services at the beginning, and scale the project with also free or cheap services.

\subsection{Environmental Dimension}

\noindent \textbf{Regarding PPP: Have you estimated the environmental impact of the project?}

This project won't cause any environmental impact directly, it's an app to solve a quotidian problem more efficiently.
It may have an indirect impact due to the services it uses, none of them has a bad reputation on managing its resources.

\noindent \textbf{Regarding PPP: Did you plan to minimize its impact, for example, by reusing resources?}

Yes, as explained earlier in section \ref{sec:hardware}, all hardware is reused.

\noindent \textbf{Regarding Useful Life: How is currently solved the problem that you want to address (state of the art)? and how will your solution improve the environment with respect to other existing solutions?}

Any of these projects produce any waste, and my solution is no different.

\newpage
\subsection{Social Dimension}

\textbf{Regarding PPP: What do you think you will achieve -in terms of personal growth- from doing this project?}

I'll learn new technologies, tools, and techniques. Having real users using the app makes me set higher standards, giving me the motivation to do everything as good as possible.

\noindent \textbf{Regarding Useful Life: How is currently solved the problem that you want to address (state of the art)?, and how will your solution improve the quality of life (social dimension) with respect other existing solutions?}

This project is about making life easier for its users. The current solutions are very inconvenient and that's why this project was born. This topic is explained in-depth in section \ref{chap:intro}.

\noindent \textbf{Regarding Useful Life: Is there a real need for the project?}

Yes, there is. Although the problem that solves isn't critical.
