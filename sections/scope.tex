\section{Scope}
\label{sec:scope}

\subsection{Objectives}
\label{sec:objectives}
The main objective of this project is to add as many requirements as possible and improve the app in general. To be more specific, the following are the most important aspects to treat:
\begin{itemize}
    \item \textbf{Add functionalities.}
    \item \textbf{Migrate code to a JavaScript task runner or bundler.}
    \item Improve UI design.
    \item Brand design.
    \item Bug fixing.
    \item Add tests to the code.
    \item Implement Continuous Integration and Deployment.
    \item Implement users accounts.
\end{itemize}

\subsection{Risks and obstacles} \label{sec:risks}

During the development of the project, there may be risks and obstacles. These are the ones I detected at the beginning:
\begin{itemize}
    \item \textbf{No decent free option}: I don't want to spend money on the project if it doesn't pay back, by an economical income, or gained knowledge. So a highly probable risk is not finding any free service (like Hosting, Database, or Instant search engine) for a specific thing. This would mean using an alternative that provides a worse user experience or not having the feature at all.
    \item \textbf{Not enough time}: Not having time to add a substantial amount of new features or releasing them too slowly may harm users' engagement. A cause of this problem may be having to learn many technologies new to me, although extra learning time is considered in the schedule. 
    \item \textbf{App becoming forgotten}: I fear that once I finish my bachelor no one is going to publicize the app among students, this is why marketing and making the app engaging is an important topic in this project. I want this project to last several student generations.
    \item \textbf{Competence appearing}: If the UPC releases a better Atenea version, many students may stop using the app. In part, this is one of the reasons why this app should be used for students from other degrees.
    \item \textbf{Fitting everyone's needs}: Every course has it's own rules, especially from different degrees. Also, every student likes doing it in his way. It's really difficult to find one solution that fits everyone, so defining the right requirements is critical.
\end{itemize}

\subsection{Changes in the objectives}

In the end, some objectives changed from the beginning. The main objective didn't change. This is the adjustments that took place:

\begin{itemize}
    \item \textbf{Migrate code to a JavaScript task runner or bundler} was \textit{Migrate code to a JavaScript framework}. I set up a bundler instead of migrating to a framework so less refactoring was required and I could spend more time in other tasks while achieving similar results. 
    \item \textbf{Brand design} was \textit{Marketing of the app}. I trimmed the scope into designing the brand specifically.
\end{itemize}
